\documentclass[10pt,aps,twocolumn]{revtex4-2}

\usepackage{graphicx}
\usepackage{dcolumn}
\usepackage{bm}
\usepackage{mathtools}
\usepackage{amssymb}
\usepackage{subcaption}
\usepackage{kotex}
\usepackage{lipsum}
\usepackage{physics}
\usepackage[table,xcdraw]{xcolor}
\usepackage{hyperref}


%\counterwithin{figure}{section}
%\counterwithin{table}{section}
%\counterwithin{equation}{section}

\linespread{1.3}

\begin{document}


\title{서강대학교 실험물리학 III Main-Report\\ Write Experiment Title Here!}

\author{20231226 강은택}
\email{etkang03@gmail.com}
\affiliation{Department of Physics, Sogang University, Seoul 04107 Korea}

\author{담당교수 : 정명화 교수 /  담당조교 : 윤한결 조교}
\altaffiliation{실험물리학III 03분반(PHY3101-03)}


\date{\today}

\begin{abstract}
보고서는 예비, 최종보고서의 2가지 종류로 이루어진다. 예비보고서는 실험시작 전에 조교에게 프린트하여 제출한다. 최종보고서는 다음 실험 시작 전에 프린트하여 제출한다. 최종보고서는 논문형식으로 실험의 목적, 이론, 방법, 분석 및 결과, 결론, Discussion, 참고문헌 정보가 모두 담긴 것을 의미한다. fitting이 필요한 데이터의 경우 MATLAB, origin 등을 이용해 처리한다. 실험 시 정숙, 청결, 정리정돈을 유지한다. 참고문헌은 형식을 맞추어 기입해야 한다. 본문에서 인용한 부분에 맞는 mark를 하지 않으면 감점, Manual과 기타 조교가 정한 기본자료만 참고한 경우 감점.
\end{abstract}




\maketitle

\section{Purpose}

본인이 이해하고 실험을 통해 얻고자 하는 목적을 기술한다.

본인이 이해한 대로 목적을 적는다. 실험 매뉴얼 그대로 적은 경우 감점.


\section{Background}

실험과 관련된 이론을 정리하여 적는다. 실험과 관련된 꼭 필요한 이론을 요약하여 적는다. Manual 그대로 쓴 경우 감점, 내용은 들어갔으나 정리되지 않은 경우 감점.

\section{Equipments and Methods}

실험에 대한 방법을 읽고 원리를 이해하여 순서대로 적는다. 읽고 이해하여 요약해 적어야 한다. 매뉴얼대로 적은 경우 감점.

\section{Data}

표와 그래프는 측정한 data를 정리하여 나타낸다. 이론과 비교 시 그래프의 정리 및 fitting이 되어야 한다.

실험별로 data가 빠짐없이 있는가? 없다면 빠진 data당 감점. 


\section{Data Analysis}
실험의 측정과 관찰을 통해 얻은 결과를 분석하고, 결론을 적는다.

data 정리가 잘 되었는가? 그래프 표의 캡션이 올바르지 않거나 없는 경우 감점, 그래프에서 좌표축 설명이 없는 경우 감점, 그래프에서 좌표축 단위가 틀린 경우 감점, 표에서 각 행, 열의 이름이 없는 경우 감점, 그래프나 표에서 계산식이 있는 경우 계산과정을 밝히지 않은 경우 감점, 반복실험의 경우 오차처리를 하지 않으면 감점.


\section{Discussion}

오차의 원인을 분석하여 적는다. 추가적으로 실행할 수 있는 실험이 있다면 제시한다. 실험에서 개선할 점이 있다면 적는다. 실험 결과가 이론과 비교하여 차이가 있다면 원인을 분석하여 기술한다.

실험전반의 요약, 측정 및 관찰을 통해 얻은 결과를 분석한다. 데이터 내용의 종합적인 해석이 아닌 단순 나열은 감점. 실험 전 계획 단계에서 예상한 점과 다르다면 이유는 무엇인지 적는다.

실험과정에서 발생한 오차를 분석한다. 향후 실험에 대한 계획을 제시한다. 다른 실험 방법에 대한 논의를 해본다. 이론적 고찰을 해본다. 단순한 감상을 적으면 감점.

\onecolumngrid

\bibliographystyle{plain}
\bibliography{references}


\nocite{*}


\end{document}