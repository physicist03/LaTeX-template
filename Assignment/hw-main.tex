\documentclass[11pt]{article}

\usepackage{graphicx}
\usepackage{amssymb}
\usepackage{mathtools}
\usepackage{amsthm}
\usepackage[font=small,labelfont=bf]{caption}
\usepackage{geometry}
\usepackage{subcaption}
\usepackage{booktabs}
\usepackage{siunitx}
\usepackage{kotex}
\usepackage{hyperref}

\geometry{a4paper, left=30mm, right=30mm, top=30mm, bottom=40mm}


\theoremstyle{definition}
\newtheorem{theorem}{Theorem}
\newtheorem{definition}{Definition}
\newtheorem{example}{Example}
\newtheorem{lemma}{Lemma}
\newtheorem{axiom}{Axiom}
\newtheorem{remark}{Remark}
\newtheorem{problem}{Problem}
\newtheorem{exercise}{Exercise}

\linespread{1.25}

\begin{document}

\begin{center}    
\huge{\textbf{Lecture Title Homework}}
\end{center}

\vspace{1mm}
\begin{flushright}
    Professor : Gil-dong Hong
\end{flushright}

\vspace{2mm}
\normalsize


\renewcommand{\theproblem}{1.23}
\begin{problem}
   여기에 과제 문제를 입력해 주세요.
\end{problem}

\noindent
\textbf{Solution:}





\end{document}