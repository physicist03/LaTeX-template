\documentclass{article}

\usepackage{kotex}
\usepackage{graphicx}
\usepackage[affil-it]{authblk}
\usepackage{mathtools}
\usepackage{amssymb}
\usepackage{geometry}
\usepackage{fancyhdr}
\usepackage{braket}
\usepackage{cite}
\usepackage{cancel}
\usepackage{subcaption}
\usepackage{enumitem}
\usepackage{xcolor}
\usepackage[most]{tcolorbox}
\usepackage{chemformula}
\usepackage{physics}
\usepackage{hyperref}

\newcommand{\vp}{\varphi}
\newcommand{\ve}{\varepsilon}

\tcbuselibrary{theorems,skins,breakable}

% Theorem
\newtcbtheorem[number within=section]{theorem}{Theorem}%
{enhanced, breakable,
 colback=white!5, colframe=cyan!70!black,
 fonttitle=\bfseries}{th}

% Definition
\newtcbtheorem[use counter from=theorem]{definition}{Definition}%
{enhanced, breakable,
 colback=white!5, colframe=blue!50!green,
 fonttitle=\bfseries}{def}

% Example
\newtcbtheorem[use counter from=theorem]{example}{Example}%
{enhanced, breakable,
 colback=white!5, colframe=black!80,
 fonttitle=\bfseries}{ex}

% Proof
\newtcolorbox{proof}[1][]{enhanced, breakable,
  colback=white!5, colframe=gray!80!black,
  fonttitle=\bfseries, title=Proof, #1}

% Lemma
\newtcbtheorem[use counter from=theorem]{lemma}{Lemma}%
{enhanced, breakable,
 colback=white!5, colframe=purple!75,
 fonttitle=\bfseries}{lem}

% Problem
\newtcbtheorem[use counter from=theorem]{problem}{Problem}%
{enhanced, breakable,
 colback=white!5, colframe=yellow!55!black,
 fonttitle=\bfseries}{prob}

% Exercise
\newtcbtheorem[use counter from=theorem]{exercise}{Exercise}%
{enhanced, breakable,
 colback=white!5, colframe=yellow!55!black,
 fonttitle=\bfseries}{exr}

% Axiom
\newtcbtheorem[use counter from=theorem]{axiom}{Axiom}%
{enhanced, breakable,
 colback=white!5, colframe=red!75!white,
 fonttitle=\bfseries}{ax}

\counterwithin{equation}{section}


\geometry{a4paper,left=2cm,right=2cm,top=2.4cm,bottom=2.4cm}

\linespread{1.3}

\title{\textsf{Write title here}}
\author[1]{Written by Eun Taek Kang\thanks{email: etkang03@gmail.com}}
\affil[1]{Department of Physics, Sogang University, Seoul 04107, Korea}

\date{Summer 2025, Sogang University}

\begin{document}

\pagestyle{fancy}
    %... then configure it.
    \fancyhf{}
    % Set the header and footer for Even
    % pages but omit the zone (L, C or R)
    \fancyhead[R]{\textsf{Prof.\ Hyun Cheol Lee}}
    \fancyhead[L]{\textsc{Course Full Name}}
    \fancyfoot[C]{\thepage}
    \fancyfoot[L]{\textbf{Sogang University}}
    \fancyfoot[R]{\textit{Department of Physics}}

\maketitle

\begin{abstract}
    여기에 적절한 서문을 작성해 주세요. 추가로 필요한 package는 자유롭게 추가해 주세요.
\end{abstract}

\newpage

\section{여기에 섹션 제목을 입력해 주세요.}

\begin{theorem}{}{}
Let $D$ be a regular domain in an oriented $n$-dimensional manifold $M$, 
and let $\omega$ be a smooth $(n-1)$ form of compact support. Then
\[\int_D d\omega = \int_{\partial D} \omega.\]
\end{theorem}

\begin{theorem}{피타고라스 정리}{pyth}
$a^2+b^2=c^2$.
\end{theorem}

\begin{definition}{정의 예시}{def1}
집합 $A$의 멱집합은 $A$의 모든 부분집합의 집합이다.
\end{definition}

\begin{example}{간단한 예}{ex1}
$A=\{1,2\}$일 때, 멱집합은 $\{\varnothing,\{1\},\{2\},\{1,2\}\}$.
\end{example}

\begin{proof}
삼각형의 면적 비교로 쉽게 증명할 수 있다.
\end{proof}

\begin{lemma}{보조정리}{lem1}
만약 $a|b$ 그리고 $b|c$이면, $a|c$이다.
\end{lemma}

\begin{problem}{문제}{prob1}
$f(x)=x^2$의 도함수를 구하라.
\end{problem}

\begin{exercise}{연습문제}{exr1}
다음 극한을 구하시오: $\operatorname{Res}_{x\to 0} \dfrac{\sin x}{x}$.
\end{exercise}

\begin{axiom}{공리}{ax1}
임의의 직선 위의 두 점을 지나는 직선이 유일하게 존재한다.
\end{axiom}


\end{document}